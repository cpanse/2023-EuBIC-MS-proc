\section{Poster Presentations}


Twenty-six posters took center stage, accompanied by dynamic 60-second flash talk presentations. One event's highlight was the election of the poster price winner (see Figure~\ref{fig:poster}. The jury consists of the five keynote speakers.


\paragraph{Tanja Holstein - Probability based taxonomic profiling of viral and microbiome samples using PepGM and Unipept}

Taxonomic inference in mass spectrometry-based metaproteomics is complex. The exact composition of metaproteomic samples is usually unknown, requiring large reference databases for peptide identification. The presence of proteins and corresponding taxa is then inferred from the identified peptides, which is complicated by protein homology: many proteins do not only share peptides within a single taxon but also between different taxa. Current taxonomic profiling approaches rely on strategies such as peptide-spectrum-match counting or the use of unique peptides. The metaproteomic analysis tool Unipept relies on a precomputed database of tryptic peptides and performs taxonomic identifications based on a threshold of peptide-taxon matches. It does not provide confidence estimates for its taxonomic assignments and restricts its results to lowest common ancestor level.
We present PepGM, a graphical model that uses belief propagation to compute the marginal distributions of peptides and taxa. Developed for the probability-based taxonomic identification of viral samples against a large reference database background, we extend it to the analysis of metaproteomic samples. Taxonomic information for peptides identified is now retrieved using the database provided by Unipept, which was extended with functionalities to restrict the taxa queried and provide all potential taxonomic origins of the peptides queried.  As a multitude of taxa may be present, this streamlines the process of obtaining the peptide-taxon relationships required by PepGM.
We evaluate the combination of PepGM and Unipept using viral and metaproteomic datasets. We show that our approach results in taxonomic profiles for metaproteomes with statistically computed confidence values. To address the complexity of the taxonomic composition of metaproteomes, results can be aggregated at different taxonomic levels. PepGM enables a robust taxonomic analysis of metaproteomic samples supporting statistically sound inference of taxa in mass spectrometric datasets. This eliminates the need for error-prone heuristics, providing the added benefit of confidence estimates for taxonomic assignments and improved taxonomic resolution.

\begin{figure}[h]
\centering
\includegraphics[width=0.3\textwidth]{images/Tanja}
\caption{The winner of the poster presentations and the poster price - Tanja Holstein Federal Institute of Materials Research and Testing (BAM), Germany; now with the VIB. source: \url{https://twitter.com/HolsteinTanja/status/1616410121600483328}.}
\label{fig:poster}
\end{figure}
